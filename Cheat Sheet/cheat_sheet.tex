\documentclass[10pt,landscape, a4paper]{article}
\usepackage{multicol}
\usepackage{calc}
\usepackage{setspace}
\usepackage{ifthen}
\usepackage[landscape]{geometry}
\usepackage{graphicx}
\usepackage{listings}

\geometry{top=.2in,left=.2in,right=.2in,bottom=.2in} 
\usepackage{lmodern}
\usepackage{listings}
\lstset{language=C,
numberstyle=\footnotesize,
basicstyle=\footnotesize,
stepnumber=1,
breaklines=true}
%\usepackage[subsection]{placeins}
\usepackage{float}

\pagestyle{empty}

\lstset{basicstyle=\footnotesizev,breaklines=true}
\lstset{framextopmargin=50pt}

\makeatletter
\renewcommand{\section}{\@startsection{section}{1}{0mm}%
                                {-1ex plus -.5ex minus -.2ex}%
                                {0.5ex plus .2ex}%x
                                {\normalfont\large\bfseries}}
\renewcommand{\subsection}{\@startsection{subsection}{2}{0mm}%
                                {-1explus -.5ex minus -.2ex}%
                                {0.5ex plus .2ex}%
                                {\normalfont\normalsize\bfseries}}
\renewcommand{\subsubsection}{\@startsection{subsubsection}{3}{0mm}%
                                {-1ex plus -.5ex minus -.2ex}%
                                {1ex plus .2ex}%
                                {\normalfont\small\bfseries}}
\makeatother

\def\BibTeX{{\rm B\kern-.05em{\sc i\kern-.025em b}\kern-.08em
    T\kern-.1667em\lower.7ex\hbox{E}\kern-.125emX}}

\setcounter{secnumdepth}{0}

\setlength{\parindent}{0pt}
\setlength{\parskip}{0pt plus 0.5ex}

\newcommand{\mysinglespacing}{%
  \setstretch{1}% no correction afterwards
}

\lstset{
         basicstyle=\footnotesize\ttfamily, 
         numberstyle=\tiny,          
         numbersep=2pt,             
         tabsize=2,                
         extendedchars=true,      
         breaklines=true,        
         showspaces=false,      
         showtabs=false,       
         xleftmargin=0pt,
         framexleftmargin=0pt,
         framexrightmargin=0pt,
         framexbottommargin=0pt,
         showstringspaces=false 
         basicstyle=\tiny
 }

\lstloadlanguages{
         C
 }

\begin{document}



% Using Courier font
\renewcommand{\ttdefault}{cmtt}

\raggedright
\footnotesize
\begin{multicols}{3}

\setlength{\premulticols}{1pt}
\setlength{\postmulticols}{1pt}
\setlength{\multicolsep}{1pt}
\setlength{\columnsep}{2pt}

\begin{center}
     \Large{\textbf{Systems Programming Cheat Sheet}} \\
\end{center}

\section{Mutex locks}

A \emph{mutex} is a lock that we set before using a shared resource and release after using it. 
When the lock is set, \emph{no other thread can access the locked region of code}. 
So we see that even if thread 2 is scheduled while thread 1 was not done accessing 
the shared resource and the code is locked by thread 1 using mutexes then thread 2 
cannot even access that region of code. So this \emph{ensures a synchronized access of shared resources in the code}.

\lstinputlisting[language=C]{snippets/mutex.c}

\section{Signal handling}

A {\bf signal} is a condition that may be reported during program execution, 
and can be ignored, handled specially, or, as is the default, 
used to terminate the program.

\lstinputlisting[language=C]{snippets/signal.c}

\section{Multiprogramming}

\subsection{Fork}

The \texttt{fork()} system call will spawn a new child process which is an 
identical process to the parent except that has a new system process ID. 
The process is copied in memory from the parent and a new process structure 
is assigned by the kernel. The return value of the function is which 
discriminates the two threads of execution. A zero is returned by the 
fork function in the child's process.

The environment, resource limits, umask, controlling terminal, 
current working directory, root directory, signal masks and other 
process resources are also duplicated from the parent in the forked 
child process.

\lstinputlisting[language=C]{snippets/fork.c}

\subsection{Exec}

The \texttt{exec()} family of functions will initiate a program from within a program. 
They are also various front-end functions to \texttt{execve()}.

The functions return an integer error code.

The function call \texttt{execl()} initiates a new program in the same environment 
in which it is operating. An executable (with fully qualified path. i.e. \texttt{/bin/ls}) 
and arguments are passed to the function. Note that \texttt{arg0} is the command/file 
name to execute.

\lstinputlisting[language=C]{snippets/exec.c}

\subsection{Wait}

\texttt{wait()}: Blocks calling process \emph{until the child process terminates}. 
If child process has already terminated, the wait() call returns immediately. 
if the calling process has multiple child processes, the function returns when one returns.

\texttt{waitpid()}: Options available to block calling process for a particular 
child process not the first one.

\section{Shell scripting}
\section{Shared memory}
\section{Pointers to functions}

\lstinputlisting[language=C]{snippets/funcpoint.c}

\section{Condition variables}
\emph{Condition variables} provide yet another way for threads to synchronize. 
While mutexes implement synchronization by controlling thread access to data, 
condition variables \emph{allow threads to synchronize based upon the actual value of data}.

Without condition variables, the programmer would need to have threads 
continually polling (possibly in a critical section), to check if the 
condition is met. This can be very resource consuming since the thread 
would be continuously busy in this activity. \emph{A condition variable is a 
way to achieve the same goal without polling.}

A condition variable is \emph{always used in conjunction with a mutex lock}.

\section{Deadlock}



\section{Semaphores}

A \emph{semaphore} is a special type of variable that can be incremented or decremented, 
but \emph{crucial access to the variable is guaranteed to be atomic}, even in a multi-threaded program.
If two or more threads in a program attempt to change the value of a semaphore, 
the system guarantees that all the operations will in fact \emph{take place in sequence}.

\end{multicols}
\end{document}
