\documentclass{article}

\title{Tokenizer Readme}
\date{\today}
\author{Paul Jones and Andrew Moore \\ Systems Programming (01:198:214) \\ School of Arts and Sciences \\ Rutgers University}

\begin{document}

\maketitle

\clearpage

\section{Description}

The program makes sure you have the correct number of arguments.
Having assured that, it allocates the necessary space and sets it all to zero.
Then it compares the separators against the current tokens.
This creates the TokenizerT structure.
Looping over this structure, every token is printed using TKGetNextToken,
assuring that a space or newline is printed appropriately.
Finally, the allocated space is freed.

\section{Features}

Instead of seeing my test cases in a boring txt file, you can
actually confirm that they all work by running the bash script
included with this assignment. There are 15 test cases, and the
script will tell you whether the code passes or fails. If you're
feeling more old school, you can go in and look at the "actual"
and "expected" variables to see what the cases are and what I expect.

\end{document}
